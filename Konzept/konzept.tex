\documentclass[a4paper, 12pt]{report}

\usepackage[ngerman]{babel} % Silbentrennung
\usepackage[utf8]{inputenc} % Umlaute
\usepackage{graphicx}

\pagestyle{headings}


\begin{document}

%%%%%%%%%%%%%%%%%%%%%%%%%%%%%%%%%%%%%%%%%%%%%%%%%%%%%%%%%%%%%%%%%%%%%%%%%%%%%

% Deckblatt
% Fehlt noch WS 13/14

\begin{titlepage}

\begin{center}

\includegraphics[width=1.0\textwidth]{fhkoeln.jpg}
\\[2cm]
\textsc{\LARGE EIS}
\\[0.2cm]
{\Large Entwicklung Interaktiver Systeme}
\\[3cm]
\textsc{\Huge Konzept}

\vfill

\textsc{\Large Team}\\
Duc Duy Khuong (11084720)\\
Robert Kellermann (11082910)
\\[1cm]
\textsc{\Large Betreuer}\\
Prof. Dr. Kristian Fischer\\
Prof. Dr. Gerhard Hartmann\\
Renée Schulz\\
Christopher Messner
\\[2cm]
\today

\end{center}

\end{titlepage}

\tableofcontents

\newpage

%%%%%%%%%%%%%%%%%%%%%%%%%%%%%%%%%%%%%%%%%%%%%%%%%%%%%%%%%%%%%%%%%%%%%%%%%%%%%

\chapter{Einführung}

% hier evtl noch einführung, problemstellung, erwartungen ka

Musik spielt bei vielen Menschen im alltäglichen Leben eine wichtige Rolle. Ob man Musik hört oder selber ein Instrument spielt, ist da jedem selber überlassen. Diejenigen, die sich sich für letzteres entscheiden, wollen meistens dann auch mit anderen Musikern zusammen Projekte starten und gründen eine Band.

Die Organisation dabei spielt eine übergeordnete Rolle, denn z.B. die Proben, hängen vom Zeit- und Ortfaktor ab, wann hat jeder Zeit und wo wird der Bandraum sein. Sind diese Fragen geklärt besteht die Absicht darin, so effizient wie möglich zu arbeiten, so dass man auch gute Resultate erzielt. 
Wenn die Probe dann zu Ende ist, ergibt sich dann die Schwierigkeit, dass vielleicht einer aus der Band spontan eine Idee hat für einen neuen Song oder ein Arrangement, aber erst warten muss, bis zur nächsten Probe, wo er dies mit den anderen teilen kann.

Aus dem Grund wäre es hilfreich, ein System zu besitzen, welches dem Musiker erlaubt, von sich zu Hause oder an einem beliebigen Ort seine Idee festzuhalten und mit den Bandkollegen zu teilen. Das System sollte also Funktionen bereitstellen, in einer bestimmten musikalischen Notation oder nach einen bestimmten Standard, z.B. MIDI, Ideen festzuhalten und diese mit den anderen teilen, so dass diese sich ein Bild davon machen können. Zusätzlich sollte der kollaborative Aspekt im Vordergrund stehen, wie es bei einer normalen Bandprobe auch der Fall ist. Demnach sollte es möglich sein, über das System miteinander zu kommunizieren, sei es durch einen Chat oder Notizen etc.. Damit die Zusammenarbeit auch funktionieren, müssen die Bandmitglieder benachrichtigt werden, damit diese stets Bescheid wissen, wenn jemand eine neue Idee hat.

Zusammengefasst soll ein System entwickelt werden, welches es ermöglicht, kollaborativ an musikalischen Projekten zu arbeiten, ohne dass man zeitlich oder örtlich gebunden ist.

%%%%%%%%%%%%%%%%%%%%%%%%%%%%%%%%%%%%%%%%%%%%%%%%%%%%%%%%%%%%%%%%%%%%%%%%%%%%%

\chapter{Ziele}

zu wenig Ziele definiert
ausformulieren, detaillierter, mehr eingehen und verschärfen
Kollaborativ ist zu wolkig -> mehr spezifizieren
zu ungenau: Begründung warum MIDI
geplante Funktionalitäten: zu technisch
Zielpriorisierung textuell formulieren, zu ungenau

\section{Strategische Ziele}

\section{Taktische Ziele}

\section{Operative Ziele}

\section[Geplante Funktionalitäten]{Geplante Funktionalitäten des Systems}

%%%%%%%%%%%%%%%%%%%%%%%%%%%%%%%%%%%%%%%%%%%%%%%%%%%%%%%%%%%%%%%%%%%%%%%%%%%%%

\chapter{Mensch-Computer-Interaktion}

zu allgemein, nicht projektbezogen, besser darlegen
Stakeholderanalyse noch spezifischer, Grundlage für Begründungen
wirkt vorgegriffen/zusammengewürfelt

%%%%%%%%%%%%%%%%%%%%%%%%%%%%%%%%%%%%%%%%%%%%%%%%%%%%%%%%%%%%%%%%%%%%%%%%%%%%%

\chapter{Kommunikationsablauf}

Datenmodell unbegründet, umstrukturieren
von Kommunikationsmodell zur Systemarchitektur führen
Kommunikationsmodell abstrakter (wer mit wem, welche Inhalte, syn/asyn, welche Wege/Alternativen)
Systemarchitektur konkreter, etwaige Softwarekomponenten, Logik der Komponenten (abstrakt)
technisch zu konkret, auf einer höheren Ebene argumentieren (Echtzeitkommunikation statt Chatroom)
Proof of Concepts abstrakter
Paradigmen kommen aus der Architektur nicht deutlich hervor, Paradigmen nicht mischen

%%%%%%%%%%%%%%%%%%%%%%%%%%%%%%%%%%%%%%%%%%%%%%%%%%%%%%%%%%%%%%%%%%%%%%%%%%%%%

\chapter{Systemarchitektur}

%%%%%%%%%%%%%%%%%%%%%%%%%%%%%%%%%%%%%%%%%%%%%%%%%%%%%%%%%%%%%%%%%%%%%%%%%%%%%

\chapter{Datenmodell}

%%%%%%%%%%%%%%%%%%%%%%%%%%%%%%%%%%%%%%%%%%%%%%%%%%%%%%%%%%%%%%%%%%%%%%%%%%%%%

\chapter{Proof-of-Concepts}

%%%%%%%%%%%%%%%%%%%%%%%%%%%%%%%%%%%%%%%%%%%%%%%%%%%%%%%%%%%%%%%%%%%%%%%%%%%%%

\chapter{Marktrecherche}

Kein Mehrwert, kein Bezug zum Projekt
Alleinstellungsmerkmale sind zu wenig

%%%%%%%%%%%%%%%%%%%%%%%%%%%%%%%%%%%%%%%%%%%%%%%%%%%%%%%%%%%%%%%%%%%%%%%%%%%%%

\chapter{Geschäftsmodell}

glaubhaft darstellen
schon Entscheidung treffen
Etablierung am Markt und Nutzerbasis fehlt

%%%%%%%%%%%%%%%%%%%%%%%%%%%%%%%%%%%%%%%%%%%%%%%%%%%%%%%%%%%%%%%%%%%%%%%%%%%%%

\chapter{Alleinstellungsmerkmal}

%%%%%%%%%%%%%%%%%%%%%%%%%%%%%%%%%%%%%%%%%%%%%%%%%%%%%%%%%%%%%%%%%%%%%%%%%%%%%

\chapter{Risiken}

alle Risiken betrachten (projektinterne Risiken)
zu grob formuliert, Lösungsvorschläge fehlen
Keine Struktur

%%%%%%%%%%%%%%%%%%%%%%%%%%%%%%%%%%%%%%%%%%%%%%%%%%%%%%%%%%%%%%%%%%%%%%%%%%%%%

\chapter{Projektplan}

Feingranularer
Stunden schätzen
genau einteilen wer was macht

%%%%%%%%%%%%%%%%%%%%%%%%%%%%%%%%%%%%%%%%%%%%%%%%%%%%%%%%%%%%%%%%%%%%%%%%%%%%%

\end{document}
