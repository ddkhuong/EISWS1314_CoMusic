
% Formale Gestaltung
	% siehe Expose & Konzept
	% sauber/lesbar
	% Artefakte in den Anhang
	% KEIN totes Wissen
	% Inhalte des Konzepts überarbeiten
	% sprachliche Präzision/Fachsprache
	% Projektspezifische Details nicht vergessen!

% Prozessdokumentation
	% Zeitlicher Verlauf des Projektes
	% Schwierigkeiten/Probleme => Behebung/Umgang
	% Wichtige Entscheidungen
	% Beschreibung der durchgeführten MCI Methoden mit ihren Ergebnissen
	% Aktivitätenverteilung (Wer hat was gemacht?)
	% Projektplan (geplant <=> tatsächlich)
	% Ziel: nachvollziehbare Prozessqualität
	
% Systemdokumentation
	% Klare Beschreibung des entwickelten Systems
	% Konzeptbestandteile:
		% Zielsetzung
		% Benutzer(gruppen)/Stakeholder
		% funktionale/qualitative/organisationale Anforderungen
		% aussagekräftiges Architekturmodell
		% technische Umsetzung
		% etc.
	% Alternativen erwähnen
	% Abwägungen
	% Begründungen
	% Abweichungen vom Konzept anführen und begründen
	% Elementare Klassen & Methoden
	% Fokus/roter Faden erkenn- und nachvollziehbar

% WICHTIG: extra Kapitel zur Darlegung der Überarbeitung des Konzepts

% Installationsdokumentation
	% Anforderungen an ein System
	% Installationsschritte: Was muss Betreiber/Admin/etc. tun?
	
% MCI und WBA2 Aspekte

	% MCI
Für ein erfolgreiches Bearbeiten der Perspektive Mensch-Computer Interaktion ist es notwendig, sich u.a. mindestens mit folgenden Fragen zu befassen, Antworten zu entwickeln, Alternativen und notwendige Kompromisse zu diskutieren und kritisch zu bewerten:

Was sind die Entwicklungsziele für das interaktive System?
Wer sind die Benutzer?
Was sind die user needs?
Was sind die funktionalen und qualitativen Anforderungen; welche Rahmenbedingungen existieren?
Welche organisatorischen Anforderungen liegen vor?
Wie soll methodisch im Projekt vorgegangen werden? Wie werden die beiden Perspektiven ("MMA" und "MCI") in EINEM Projekt systematisch und strukturiert berücksichtigt? Vorgehensmodelle? Begründung und kritische Diskussion der Alternativen notwendig.
Welches sind die Aufgaben, welche Struktur weisen sie auf und in welchen Beziehungen stehen die Aufgaben zueinander?
Wie sieht ein valides Modell des Nutzungskontextes aus?
Welche Metaphern und Paradigmen liegen in der Domäne vor?
Welches ist das konzeptuelle Modell für das präskriptive Aufgabenmodell? Welche Bezüge existieren bzgl. der Entwicklungsziele?
Welche alternativen konzeptuellen Modelle wurden entwickelt und wie wurde mit Design-Alternativen verfahren?
Welche Interaktions-Paradigmen, -Modi und -Stile wurden in Betracht gezogen? Erörterung notwendig.
Welche Prototypen wurden entwickelt, was zeichnet die verschiedenen Prototypen-Alternativen aus und wie wurde eine Synthese der Alternativen erreicht?
Nach welchen Kriterien wurden die Evaluation-Methoden und -Techniken ausgewählt?
Welches sind die Evaluationsergebnisse, welchen Erfüllungsgrad der Entwicklungsziele weist der finale Prototyp auf?
Welche Konsequenzen für das Redesign ergeben sich?
Wie sehen die weiteren Iterationen aus?
Diese Liste ist von Ihnen für Ihr Projekt zu vervollständigen.	

	% WBA2
Für ein erfolgreiches Bearbeiten der Perspektive der Architektur des Systems auf Grundlage der in der Veranstaltung Web-basierte Anwendungen 2: Verteilte Systeme vermittelten Konzepte ist eine Analyse und Diskussion unter anderem entlang der im Folgenden aufgeführten Fragen erforderlich. Bei der Diskussion sollte die Abwägung primär durch eine Abwägung der Erkenntnisse aus der MCI Perspektive und durch der Eigenschaften von zugänglichen Software Rahmenwerken motiviert sein.

Aus welchen Komponenten besteht das System und welche Interaktion findet zwischen den Komponenten statt?
Sind die Komponenten auf räumlich verteilten Systemen angesiedelt?
Werden unterschiedliche Arten von Endgeräten wie etwa mobile Geräte und Arbeitsplatzsysteme unterstellt?
Welche der Interaktionen sind synchron und welche asynchron?
Welche Daten werden in den einzelnen Interaktionsschritten ausgetauscht?
Welche Middlewaresysteme sind geeignet, die erforderliche Interaktion zu unterstützen?
In welcher Sprache und in welchem Typsystem werden die Daten modelliert und wie werden sie persistent gespeichert?
Welche Funktionalitäten bzw. Interaktionen sind wesentlich, um den Kern des Systems demonstrieren zu können?

% Projektplan
	% Hoher Detaillierungsgrad
		% Voraus planen!
			% Iterationen/vorgegebene und eigene Meilensteine definieren
			% Vorgehensmodell berücksichtigen/in Literatur Methoden, Techniken, Gestaltungsaktivitäten aus der ISO... nachschlagen
			% Pufferzeiten
			% Architektur und PoC berücksichtigen
	% Aufteilung der Arbeitszeit zwischen den Gruppenmitgliedern
	% Geplanter Aufwand vs. eingetretener Aufwand in Stunden
	% Wichtig: Anhand der Zielpriorisierung bei Zeitmangel begründetes Weglassen einzelner Punkte

% Allgemeines Fazit
	% Erfüllungsgrad
	% Bewertung des eigenen Prozesses