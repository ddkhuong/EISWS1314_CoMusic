\documentclass[12pt]{scrartcl}

\usepackage[T1]{fontenc}
\usepackage[ngerman]{babel} % Silbentrennung
\usepackage[utf8]{inputenc} % Umlaute
\usepackage{graphicx}
\usepackage[style=authortitle-icomp]{biblatex}
\usepackage[babel,german=guillemets]{csquotes}

\usepackage{xcolor}
\definecolor{dark-red}{rgb}{0.4,0.15,0.15}
\definecolor{dark-blue}{rgb}{0.15,0.15,0.4}
\definecolor{medium-blue}{rgb}{0,0,0.5}

\usepackage{hyperref}
\hypersetup{
    colorlinks, linkcolor={dark-red},
    citecolor={dark-blue}, urlcolor={medium-blue}
}


\begin{document}



%%%%%%%%%%%%%%%%%%%%%%%%%%%%%%%%%%%%%%%%%%%%%%%%%%%%%%%%%%%%%%%%%%%%%%%%%%%%%

% Deckblatt
% Fehlt noch WS 13/14

\begin{titlepage}

\begin{center}

\includegraphics[width=1.0\textwidth]{fhkoeln.jpg}
\\[2cm]
\textsc{\LARGE EIS}
\\[0.2cm]
{\Large Entwicklung Interaktiver Systeme}
\\[3cm]
\textsc{\Huge Konzept}

\vfill

\textsc{\Large Team}\\
Duc Duy Khuong (11084720)\\
Robert Kellermann (11082910)
\\[1cm]
\textsc{\Large Betreuer}\\
Prof. Dr. Kristian Fischer\\
Prof. Dr. Gerhard Hartmann\\
Renée Schulz\\
Christopher Messner
\\[2cm]
\today

\end{center}

\end{titlepage}

\tableofcontents

\newpage

%%%%%%%%%%%%%%%%%%%%%%%%%%%%%%%%%%%%%%%%%%%%%%%%%%%%%%%%%%%%%%%%%%%%%%%%%%%%%

\section{Einführung}

Musik spielt bei vielen Menschen im alltäglichen Leben eine wichtige Rolle. Ob man Musik hört oder selber ein Instrument spielt, ist da jedem selber überlassen. Diejenigen, die sich sich für Letzteres entscheiden, wollen dann meistens auch mit anderen Musikern zusammen Projekte starten und gründen eine Band.

Die Organisation der Band spielt dabei eine übergeordnete Rolle, denn die Bandproben hängen vom Zeit- und Ortsfaktor ab, wann hat Jeder Zeit und wo wird der Bandraum sein. Wenn diese Fragen geklärt sind, besteht die Absicht darin, die Zeit der Probe so effizient wie möglich zu nutzen.

%% vielleicht noch etwas ausführlicher?

\subsection{Problem}

Ein grundlegendes Problem ist der Umgang mit neuen Ideen. Wenn beispielsweise ein Musiker eine Idee zu einem neuen Song hat, etwa ein Riff oder eine kurze Melodie, muss er bis zur nächsten Probe warten, um sie der Band vorzustellen. Zu dieser Idee steuern die anderen Musiker dann meist ihre Ideen bei und entwickeln somit gemeinsam einen neuen Song. Dieser kreative Prozess nimmt jedoch viel Zeit in Anspruch. Aufgrund der wertvollen Zeit im Proberaum entsteht somit ein Druck auf den Bandmitgliedern und der Kreativität der Musiker wird möglicherweise kein freier Lauf gelassen.

\subsection{Idee}

Um diesen Zeitdruck beim Komponieren im Proberaum zu vermeiden, wäre es hilfreich, neue Ideen auch außerhalb der Bandproben festzuhalten und mit den Bandkollegen zu teilen. Ein Austausch von Ideen und das Beitragen von Einfällen und weiteren Ideen dazu könnte entweder alleine in Ruhe oder in Kommunikation mit den anderen Bandmitgliedern über das Internet erfolgen. Dadurch könnte die Zeit im Proberaum für das tatsächliche Proben des Zusammenspiels in der Band und den letzten Schliff an neuen Songs genutzt werden, während die kreativen, meist sehr zeitaufwändigen Kompositionsprozesse außerhalb der Proberaumatmosphäre stattfinden.

%%%%%%%%%%%%%%%%%%%%%%%%%%%%%%%%%%%%%%%%%%%%%%%%%%%%%%%%%%%%%%%%%%%%%%%%%%%%%

\section{Ziele}

zu wenig Ziele definiert
ausformulieren, detaillierter, mehr eingehen und verschärfen
Kollaborativ ist zu wolkig -> mehr spezifizieren
zu ungenau: Begründung warum MIDI
geplante Funktionalitäten: zu technisch
Zielpriorisierung textuell formulieren, zu ungenau

Strategische Ziele

Aus dem Grund wäre es hilfreich, ein System zu besitzen, welches dem Musiker erlaubt, von sich zu Hause oder an einem beliebigen Ort seine Idee festzuhalten und mit den Bandkollegen zu teilen. Das System sollte also Funktionen bereitstellen, in einer bestimmten musikalischen Notation oder nach einen bestimmten Standard, z.B. MIDI, Ideen festzuhalten und diese mit den anderen teilen, so dass diese sich ein Bild davon machen können. Zusätzlich sollte der kollaborative Aspekt im Vordergrund stehen, wie es bei einer normalen Bandprobe auch der Fall ist. Demnach sollte es möglich sein, über das System miteinander zu kommunizieren, sei es durch einen Chat oder Notizen etc.. Damit die Zusammenarbeit auch funktionieren, müssen die Bandmitglieder benachrichtigt werden, damit diese stets Bescheid wissen, wenn jemand eine neue Idee hat.

Zusammengefasst soll ein System entwickelt werden, welches es ermöglicht, kollaborativ an musikalischen Projekten zu arbeiten, ohne dass man zeitlich oder örtlich gebunden ist.


Das Hauptziel des Projektes ist es Bands zu erleichtern auch außerhalb von den Proben gemeinsam Musik machen zu können, ohne dass sich diese treffen müssen.
Taktische Ziele

Taktisches Ziel ist die Entwicklung eines Systems, das ein kollaboratives Arbeiten von Musikern ermöglicht. Dazu gehört das Erstellen und Bearbeiten von Kompositionen. Außerdem soll dies in Echtzeit möglich sein.
Operative Ziele

Um die strategischen und taktischen Ziele zu erfüllen werden zunächst geeignete Vorgehens- und Gestaltungsmethoden der Mensch-Computer-Interaktion ausgewählt. Darauf hin werden die technischen Anforderungen des Systems festgelegt und schließlich ein Prototyp entwickelt, welches die Funktionalitäten des System demonstriert.
geplante Funktionalitäten des Systems:

Band- und Musikerverwaltung (einzelne Accounts für die Musiker, gemeinsamer Bereich für die jeweiligen Bands
kollaboratives Arbeiten an MIDI-Spuren innerhalb einer Kompositionsidee
Echtzeitaktualisierung der Spuren
kompositionsbezogene Kommunikation der Bandmitglieder (Chatroom, Notizen o.Ä.)
Optional: Generierung von Notenblättern und Tabulaturen anhand der MIDI-Dateien



\subsection{Strategische Ziele}

\subsection{Taktische Ziele}

\subsection{Operative Ziele}

\subsection[Geplante Funktionalitäten]{Geplante Funktionalitäten des Systems}

%% Hier evtl zusammenfassung als tabelle

%%%%%%%%%%%%%%%%%%%%%%%%%%%%%%%%%%%%%%%%%%%%%%%%%%%%%%%%%%%%%%%%%%%%%%%%%%%%%

\section{Mensch-Computer-Interaktion}

zu allgemein, nicht projektbezogen, besser darlegen
Stakeholderanalyse noch spezifischer, Grundlage für Begründungen
wirkt vorgegriffen/zusammengewürfelt

%%%%%%%%%%%%%%%%%%%%%%%%%%%%%%%%%%%%%%%%%%%%%%%%%%%%%%%%%%%%%%%%%%%%%%%%%%%%%

\section{Kommunikationsablauf}

Kommunikationsmodell abstrakter (wer mit wem, welche Inhalte, syn/asyn, welche Wege/Alternativen)
technisch zu konkret, auf einer höheren Ebene argumentieren (Echtzeitkommunikation statt Chatroom)



%%%%%%%%%%%%%%%%%%%%%%%%%%%%%%%%%%%%%%%%%%%%%%%%%%%%%%%%%%%%%%%%%%%%%%%%%%%%%

\section{Systemarchitektur}

Paradigmen kommen aus der Architektur nicht deutlich hervor, Paradigmen nicht mischen
von Kommunikationsmodell zur Systemarchitektur führen
Systemarchitektur konkreter, etwaige Softwarekomponenten, Logik der Komponenten (abstrakt)

Vorgabe: java

%%%%%%%%%%%%%%%%%%%%%%%%%%%%%%%%%%%%%%%%%%%%%%%%%%%%%%%%%%%%%%%%%%%%%%%%%%%%%

\section{Datenmodell}

Datenmodell unbegründet, umstrukturieren

%%%%%%%%%%%%%%%%%%%%%%%%%%%%%%%%%%%%%%%%%%%%%%%%%%%%%%%%%%%%%%%%%%%%%%%%%%%%%

\section{Proof-of-Concepts}

Proof of Concepts abstrakter

%%%%%%%%%%%%%%%%%%%%%%%%%%%%%%%%%%%%%%%%%%%%%%%%%%%%%%%%%%%%%%%%%%%%%%%%%%%%%

\section{Marktrecherche}

Kein Mehrwert, kein Bezug zum Projekt
Alleinstellungsmerkmale sind zu wenig

%%%%%%%%%%%%%%%%%%%%%%%%%%%%%%%%%%%%%%%%%%%%%%%%%%%%%%%%%%%%%%%%%%%%%%%%%%%%%

\section{Geschäftsmodell}

glaubhaft darstellen
schon Entscheidung treffen
Etablierung am Markt und Nutzerbasis fehlt

%%%%%%%%%%%%%%%%%%%%%%%%%%%%%%%%%%%%%%%%%%%%%%%%%%%%%%%%%%%%%%%%%%%%%%%%%%%%%

\section{Alleinstellungsmerkmal}

%%%%%%%%%%%%%%%%%%%%%%%%%%%%%%%%%%%%%%%%%%%%%%%%%%%%%%%%%%%%%%%%%%%%%%%%%%%%%

\section{Risiken}

alle Risiken betrachten (projektinterne Risiken)
zu grob formuliert, Lösungsvorschläge fehlen
Keine Struktur

%% PUNKTE
- stehende internetverbindung vorrausgesetzt

%%%%%%%%%%%%%%%%%%%%%%%%%%%%%%%%%%%%%%%%%%%%%%%%%%%%%%%%%%%%%%%%%%%%%%%%%%%%%

\section{Projektplan}

Feingranularer
Stunden schätzen
genau einteilen wer was macht

%%%%%%%%%%%%%%%%%%%%%%%%%%%%%%%%%%%%%%%%%%%%%%%%%%%%%%%%%%%%%%%%%%%%%%%%%%%%%

\end{document}
