\documentclass[12pt]{scrartcl}

\usepackage[T1]{fontenc}
\usepackage[ngerman]{babel} % Silbentrennung
\usepackage[utf8]{inputenc} % Umlaute
\usepackage{graphicx}
\usepackage[style=authortitle-icomp]{biblatex}
\usepackage[babel,german=guillemets]{csquotes}

\usepackage{xcolor}
\definecolor{dark-red}{rgb}{0.4,0.15,0.15}
\definecolor{dark-blue}{rgb}{0.15,0.15,0.4}
\definecolor{medium-blue}{rgb}{0,0,0.5}

\usepackage{hyperref}
\hypersetup{
    colorlinks, linkcolor={dark-red},
    citecolor={dark-blue}, urlcolor={medium-blue}
}


\begin{document}



%%%%%%%%%%%%%%%%%%%%%%%%%%%%%%%%%%%%%%%%%%%%%%%%%%%%%%%%%%%%%%%%%%%%%%%%%%%%%

% Deckblatt
% Fehlt noch WS 13/14

\begin{titlepage}

\begin{center}

\includegraphics[width=1.0\textwidth]{fhkoeln.jpg}
\\[2cm]
\textsc{\LARGE EIS}
\\[0.2cm]
{\Large Entwicklung Interaktiver Systeme}
\\[3cm]
\textsc{\Huge Konzept}

\vfill

\textsc{\Large Team}\\
Duc Duy Khuong (11084720)\\
Robert Kellermann (11082910)
\\[1cm]
\textsc{\Large Betreuer}\\
Prof. Dr. Kristian Fischer\\
Prof. Dr. Gerhard Hartmann\\
Renée Schulz\\
Christopher Messner
\\[2cm]
\today

\end{center}

\end{titlepage}

\tableofcontents

\newpage

%%%%%%%%%%%%%%%%%%%%%%%%%%%%%%%%%%%%%%%%%%%%%%%%%%%%%%%%%%%%%%%%%%%%%%%%%%%%%

\section{Einführung}

Musik spielt bei vielen Menschen im alltäglichen Leben eine wichtige Rolle. Ob man Musik hört oder selber ein Instrument spielt, ist da jedem selber überlassen. Diejenigen, die sich sich für Letzteres entscheiden, wollen dann meistens auch mit anderen Musikern zusammen Projekte starten und gründen eine Band.

Die Organisation der Band spielt dabei eine übergeordnete Rolle, denn die Bandproben hängen vom Zeit- und Ortsfaktor ab, wann hat Jeder Zeit und wo wird der Bandraum sein. Wenn diese Fragen geklärt sind, besteht die Absicht darin, die Zeit der Probe so effizient wie möglich zu nutzen.

%% vielleicht noch etwas ausführlicher?

\subsection{Problem}

Ein grundlegendes Problem ist der Umgang mit neuen Ideen. Wenn beispielsweise ein Musiker eine Idee zu einem neuen Song hat, etwa ein Riff oder eine kurze Melodie, muss er bis zur nächsten Probe warten, um sie der Band vorzustellen. Zu dieser Idee steuern die anderen Musiker dann meist ihre Ideen bei und entwickeln somit gemeinsam einen neuen Song. Dieser kreative Prozess nimmt jedoch viel Zeit in Anspruch. Aufgrund der wertvollen Zeit im Proberaum entsteht somit ein Druck auf den Bandmitgliedern und der Kreativität der Musiker wird möglicherweise kein freier Lauf gelassen.

\subsection{Idee}

Um diesen Zeitdruck beim Komponieren im Proberaum zu vermeiden, wäre es hilfreich, neue Ideen auch außerhalb der Bandproben festzuhalten und mit den Bandkollegen zu teilen. Ein Austausch von Ideen und das Beitragen von Einfällen und weiteren Ideen dazu könnte entweder alleine in Ruhe oder in Kommunikation mit den anderen Bandmitgliedern über das Internet erfolgen. Dadurch könnte die Zeit im Proberaum für das tatsächliche Proben des Zusammenspiels in der Band und den letzten Schliff an neuen Songs genutzt werden, während die kreativen, meist sehr zeitaufwändigen Kompositionsprozesse außerhalb der Proberaumatmosphäre stattfinden.

%%%%%%%%%%%%%%%%%%%%%%%%%%%%%%%%%%%%%%%%%%%%%%%%%%%%%%%%%%%%%%%%%%%%%%%%%%%%%

\section{Ziele}

In dieser Zielhierarchie soll deutlich gemacht werden, welche kurz- und langfristigen Ziele das Projekt \emph{CoMusic} ausmachen. Danach soll dann in der Zielpriorisierung aufgezählt werden, welche dieser Ziele die Besonderheiten und die Alleinstellungsmerkmale des Projektes darstellen.
% Einführung zu den Zielen vllt noch etwas ausführlicher? Zielpriorisierung besser erklären?

\subsection{Strategische Ziele}

% zu wenig Ziele definiert
% ausformulieren, detaillierter, mehr eingehen und verschärfen
\begin{itemize}


\item Vermeiden von Zeitdruck bei der kreativen Arbeit an neuen Songideen
\item Ideen und Einfälle sollen auch alleine in Ruhe festzuhalten sein 
\item Bandaktivitäten auch außerhalb der Proben ermöglichen
\end{itemize}

% alter text
Aus dem Grund wäre es hilfreich, ein System zu besitzen, welches dem Musiker erlaubt, von sich zu Hause oder an einem beliebigen Ort seine Idee festzuhalten und mit den Bandkollegen zu teilen. Das System sollte also Funktionen bereitstellen, in einer bestimmten musikalischen Notation oder nach einen bestimmten Standard, z.B. MIDI, Ideen festzuhalten und diese mit den anderen teilen, so dass diese sich ein Bild davon machen können. Zusätzlich sollte der kollaborative Aspekt im Vordergrund stehen, wie es bei einer normalen Bandprobe auch der Fall ist. Demnach sollte es möglich sein, über das System miteinander zu kommunizieren, sei es durch einen Chat oder Notizen etc.. Damit die Zusammenarbeit auch funktionieren, müssen die Bandmitglieder benachrichtigt werden, damit diese stets Bescheid wissen, wenn jemand eine neue Idee hat.

Das Hauptziel des Projektes ist es Bands zu erleichtern auch außerhalb von den Proben gemeinsam Musik machen zu können, ohne dass sich diese treffen müssen.

\subsection{Taktische Ziele}

Taktisches Ziel ist die Entwicklung eines Systems, das ein kollaboratives Arbeiten von Musikern ermöglicht. Dazu gehört das Erstellen und Bearbeiten von Kompositionen. Außerdem soll dies in Echtzeit möglich sein.
\begin{itemize}
\item gemeinsam nutzbare Plattform
\item Das Teilen und Ausarbeiten von Songideen soll in Ruhe außerhalb des Proberaumes geschehen können
\item Unabhängiges Arbeiten voneinander
\item Kommunizieren der Bandmitglieder über Ideen soll möglich sein
\
\end{itemize}
% Kollaborativ ist zu wolkig -> mehr spezifizieren

\subsection{Operative Ziele}

Um die strategischen und taktischen Ziele zu erfüllen werden zunächst geeignete Vorgehens- und Gestaltungsmethoden der Mensch-Computer-Interaktion ausgewählt. Darauf hin werden die technischen Anforderungen des Systems festgelegt und schließlich ein Prototyp entwickelt, welches die Funktionalitäten des System demonstriert.

\begin{itemize}
\item Vorgehensmodelle und Gestaltungsmethoden der MCI auswählen
\item Festlegung der technischen Anforderungen
\item Entwicklung eines Prototyps 
\end{itemize}

\subsection[Geplante Funktionalitäten]{Geplante Funktionalitäten des Systems}

\begin{itemize}
\item Band- und Musikerverwaltung (einzelne Accounts für die Musiker, gemeinsamer Bereich für die jeweiligen Bands
\item kollaboratives Arbeiten an MIDI-Spuren innerhalb einer Kompositionsidee
\item Echtzeitaktualisierung der Spuren
\item kompositionsbezogene Kommunikation der Bandmitglieder (Chatroom, Notizen o.Ä.)
\item Optional: Generierung von Notenblättern und Tabulaturen anhand der MIDI-Dateien
\item ------------------------------
\item Band- und Musikerverwaltung
\item Zusammenarbeit an einer Kompositionsidee 
\item Ermöglichung der Echtzeit-Zusammenarbeit trotz örtlicher Unabhängigkeit
\item Option: Generierung von Musikblättern mit einheitlicher Musiknotation
\end{itemize}






% geplante Funktionalitäten: zu technisch
% Zielpriorisierung textuell formulieren, zu ungenau
% zu ungenau: Begründung warum MIDI

%% Hier evtl zusammenfassung als tabelle

%%%%%%%%%%%%%%%%%%%%%%%%%%%%%%%%%%%%%%%%%%%%%%%%%%%%%%%%%%%%%%%%%%%%%%%%%%%%%

\section{Mensch-Computer-Interaktion}

zu allgemein, nicht projektbezogen, besser darlegen
Stakeholderanalyse noch spezifischer, Grundlage für Begründungen
wirkt vorgegriffen/zusammengewürfelt

\subsection{Abwägung Nutzerzentrierte/Nutzungszentrierte Gestaltung}
Um eine geeignete Grundlage zu schaffen, damit eine erfolgreiche Umsetzung des System erfolgen kann, sollte eine vollständige und durchdachte Planung stattfinden. Dazu zählt die Konzeptionierung, Evaluation, Gestaltung und die Entwicklung. 
Mit Hilfe verschiedener Vorgehensmodelle und Methoden der Mensch-Computer-Interaktion ist es möglich dies zu realisieren.

Erste zentrale Frage ist es, festzustellen welche der beiden Möglichkeiten, benutzerzentrierte oder benutzungszentrierte Gestaltung, bei der Entwicklung des Systems sinnvoller ist.

Aus der Problematik bzw. der Motivation, lässt sich ableiten, dass die Entwicklung der Anwendung der Intention dient, dass die Bedüfnisse der Benutzer also die “user needs” erfüllt werden und daraus folgend eine gute “user experience” zu erreichen. 
In dem Fall sind die Benutzer, die im Hauptfokus des Projekts stehen, die Musiker, speziell die in Gruppen wie Bands etc., was bedeutet, dass das System auf die eine Zielgruppe zugeschnitten werden muss. Auch der Nutzungskontext lässt sich daraus ableiten, wodurch sich eine benutzerzentrierte Gestaltung anbietet. Die Anforderungen der Benutzer können herauskristallisiert werden wodurch das System weitesgehend angepasst werden kann.

Dadurch sollten sie möglichst auch ohne jegliche Vorkenntnisse, bis auf die in unserem spezifischen Nutzungskontext erforderten, mit dem System interagieren können.
Um dies gewährleisten zu können ist es notwendig, die Gebrauchstauglichkeit des Systems zu maximieren, d.h. die Entwicklung und Gestaltung. Alles, was ihn in der Interaktion mit dem System beeinträchtigen könnte, sollte möglichst vermieden werden. Im Gegenteil, die Anwendung sollte den Benutzer in der Verwendung unterstützen können. Dazu zählt z.B. eine nicht zu komplexe grafische Benutzeroberfläche (GUI). Die Elemente der GUI sollten so gewählt sein, dass die Gebrauchstauglichkeit hoch ist keine Verwirrung oder Überforderung stattfindet.


%%%%%%%%%%%%%%%%%%%%%%%%%%%%%%%%%%%%%%%%%%%%%%%%%%%%%%%%%%%%%%%%%%%%%%%%%%%%%

\section{Kommunikationsablauf}

Kommunikationsmodell abstrakter (wer mit wem, welche Inhalte, syn/asyn, welche Wege/Alternativen)
technisch zu konkret, auf einer höheren Ebene argumentieren (Echtzeitkommunikation statt Chatroom)



%%%%%%%%%%%%%%%%%%%%%%%%%%%%%%%%%%%%%%%%%%%%%%%%%%%%%%%%%%%%%%%%%%%%%%%%%%%%%

\section{Systemarchitektur}

Paradigmen kommen aus der Architektur nicht deutlich hervor, Paradigmen nicht mischen
von Kommunikationsmodell zur Systemarchitektur führen
Systemarchitektur konkreter, etwaige Softwarekomponenten, Logik der Komponenten (abstrakt)

Vorgabe: java

%%%%%%%%%%%%%%%%%%%%%%%%%%%%%%%%%%%%%%%%%%%%%%%%%%%%%%%%%%%%%%%%%%%%%%%%%%%%%

\section{Datenmodell}

Datenmodell unbegründet, umstrukturieren

%%%%%%%%%%%%%%%%%%%%%%%%%%%%%%%%%%%%%%%%%%%%%%%%%%%%%%%%%%%%%%%%%%%%%%%%%%%%%

\section{Proof-of-Concepts}

Proof of Concepts abstrakter

%%%%%%%%%%%%%%%%%%%%%%%%%%%%%%%%%%%%%%%%%%%%%%%%%%%%%%%%%%%%%%%%%%%%%%%%%%%%%

\section{Marktrecherche}

% Kein Mehrwert, kein Bezug zum Projekt
% Alleinstellungsmerkmale sind zu wenig

Für die Marktrecherche wurden Produkte und Dienste untersucht, die dem Konzept von CoMusic ähnlich sind und eine Konkurrenz darstellen könnten. Dabei wurde auf zwei verschiedene Kategorien von Produkten gestoßen, welche das Konzept von CoMusik vereint.

\subsection{kollaborative Kompositionsplattformen}

Die untersuchten Produkte dieser Kategorie ermöglichen einen Austausch von Samples oder anderen Beispielaufnahmen wie Gesang sowie Texten und vieles mehr. Es können Musikideen aufgenommen und hochgeladen und mit anderen Musikern geteilt werden. So kann jeder Musiker seine Ideen zu einem Projekt beisteuern und die Beteiligten kennen in etwa die einzelnen Ideen. Ein hörbares, zusammengesetztes Beispiel aller Ideen beziehungsweise Instrumente ist hier aber nicht möglich und eine genauere Struktur oder Notation aller Instrumente ist nicht ersichtlich. Änderungen an den aufgenommenen Spuren sind nicht direkt möglich und es gibt nur sehr wenig Spielraum für Experimente, ohne eine Spur komplett neu aufzunehmen.

% Bewertung hier?
Diese Art der Kollaboration eignet sich eventuell für unabhängige Musiker, welche gemeinsam rein über das Internet Kompositionen erstellen wollen, nicht aber für eine gute Vorbereitung der Mitglieder einer Band auf eine anstehende Bandprobe.

\begin{description}
\item[Kompoz]
Möglichkeit der Kollaboration mit fremden Personen
einzelne Instrumente/Vocals sind vorhanden
Ergänzung durch andere Mitglieder der Seite 

\end{description}


\subsection{Notationseditoren}






Kompoz (www.kompoz.com)


Sound Collabs (www.soundcollabs.com)
Suche nach Komponenten(z.B. “Suche Gesang für folgendes”)
Remix Anfragen, Möglichkeit eigene Songs zum Remix für andere bereitzustellen
Verbindung mit Musiklabels, welche nach Demos suchen (Möglichkeit für Bands “rauszukommen”)

v-band (www.v-band.de)
Reines Forum mit Dateiupload.

Rifflet (www.rifflet.com)
Plattform für das Weiterentwickeln von unfertigen und/oder aufgegebenen Songs Anderer.


Digital Musician (www.digitalmusician.net)
Eher kommerziell, Projektmithilfe wie Jobangebote


Die andere Kategorie sind Dienste oder Produkte, welche das Komponieren im MIDI-Format oder anderen Notationen ermöglicht. Dies bedeutet eine präzise Struktur und einen genauen Überblick über alle Spuren für alle Beteiligten. Ideen sind leicht für jeden Musiker änderbar und man hat einen großen Spielraum zum Ausprobieren und Experimentieren.
Im Bereich Editoren und Sequenzer für verschiedene Notationen (Noten, Tabulaturen) und auch für das MIDI-Format gibt es bereits unzählige Lösungen, welche teilweise einen sehr großen Funktionsumfang und eine komplexe Bedienung haben können. Die Palette reicht von einfachen Editoren wie NoteEdit (http://noteedit.berlios.de/) über mehrspurige Kompositionswerkzeuge wie GuitarPro bis hin zu professionellen Lösungen wie Cubase (http://www.steinberg.net/de/products/cubase/) oder Logic (http://www.apple.com/de/logic-pro/), welche neben den in diesem Kontext interessanten Funktionen noch viele andere Funktionen bieten. Aufgrund der Komplexität der meisten Produkte in diesem Bereich möchten wir an dieser Stelle die Webapplikation Inudge (www.inudge.net) hervorheben, welche wegen ihrer Simplizität unserem Konzept sehr nahe kommt.
Alle der eben aufgezählten Produkte haben einen gemeinsamen Nachteil: es ist keine einfache Kollaboration in Echtzeit möglich, lediglich das Teilen bereits notierter Ideen ist möglich.



%%%%%%%%%%%%%%%%%%%%%%%%%%%%%%%%%%%%%%%%%%%%%%%%%%%%%%%%%%%%%%%%%%%%%%%%%%%%%

\section{Geschäftsmodell}

% glaubhaft darstellen
% schon Entscheidung treffen
% Etablierung am Markt und Nutzerbasis fehlt

\subsection{mögliche Geschäftsmodelle}

Um ein solches System gewinnbringend zu vermarkten, gibt es mehrere Alternativen.

\paragraph{Lizenzen}
Als Erstes könnte man das Produkt, mit dem der Benutzer arbeitet, zu einem Festpreis verkaufen. Dies würde jedoch Bands abschrecken, da jeder Musiker für sich das Produkt kaufen müsste. Falls eine Nutzung doch erwünscht ist, würden aus finanziellen Gründen eventuell nur ein Teil der Musiker (etwa die Hauptkomponisten) das Produkt kaufen und das gemeinsame Komponente der gesamten Band wäre nicht mehr gegeben. Außerdem ist ein solches Geschäftsmodell unberechenbarer als Andere, da Lizenzen vergeben werden müssen, diese möglicherweise innerhalb der Band von mehreren Personen genutzt werden oder sogar für die Öffentlichkeit angeboten werden. Technische Funktionen für die Lizenzprüfung könnten durch illegale Modifikation der Software ausgehebelt werden und dem Geschäftsmodell schaden. Daher ist dieses Geschäftsmodell für unser Projekt eher weniger von Interesse.

\paragraph{Werbung}
Andererseits könnte man das Produkt zunächst kostenlos anbieten, jedoch (kontextbezogene) Werbung einblenden. Werbeflächen innerhalb eines Programm wirken meist allerdings sehr störend und abschreckend. Daher wäre das Einblenden von Werbung von Sponsoren aus dem Musikbereich beim Starten oder Beenden des Programmes von Werbung eine weitere Möglichkeit.

\paragraph{monatliche Gebühren}
Eine andere Möglichkeit wäre, für die Nutzung des Systems einer ganzen Band etwa monatlich Gebühren zu verlangen. So kann der Zugang zum Beispiel aus der Bandkasse finanziert werden und alle Mitglieder sind frei in der Nutzung. Es besteht also kein Ausgrenzen oder sonstiger Druck auf die Band und auch würde  störende Werbung kein Thema mehr sein.

\subsection{Geschäftsmodell für CoMusic}

Für dieses Projekt wurde eine Kombination der beiden letzten Möglichkeiten ausgewählt. Das Produkt soll also über monatliche Gebühren finanziert werden. Bezahlt wird als gesamte Band, sodass alle Mitglieder der Band das System nutzen können und die Nutzung beispielsweise aus der Bandkasse finanziert wird. Um das Produkt jedoch am Markt etablieren zu können und potentielle Nutzer nicht von Gebühren abzuschrecken, soll das Produkt einen kostenlosen Testzeitraum zur Verfügung stellen, in dem Bands das System testen und sich unverbindlich von CoMusic überzeugen lassen können.

Um das Risiko eines Misserfolges zu vermindern, soll während der kostenlosen Testphase beim Starten oder Beenden des Programmes dezente und kontextbezogene Werbung von Sponsoren aus dem Musikbereich angezeigt werden.


%%%%%%%%%%%%%%%%%%%%%%%%%%%%%%%%%%%%%%%%%%%%%%%%%%%%%%%%%%%%%%%%%%%%%%%%%%%%%

\section{Alleinstellungsmerkmal}

Wie bereits in der Marktrecherche deutlich geworden ist, schlägt unser Konzept eine Brücke zwischen den bisher unabhängigen Kategorien Kollaborierungsdienste und Notationseditoren. Insbesondere die einfache und die intuitive Bedienung für das kollaborative Entwickeln und Vorrantreiben von Kompositionsideen speziell für Bands als Vor- und Nachbereitung von Bandproben charakterisieren das Alleinstellungsmerkmal unseres Konzeptes.

%%%%%%%%%%%%%%%%%%%%%%%%%%%%%%%%%%%%%%%%%%%%%%%%%%%%%%%%%%%%%%%%%%%%%%%%%%%%%

\section{Risiken}

% alle Risiken betrachten (projektinterne Risiken)
% zu grob formuliert, Lösungsvorschläge fehlen
% Keine Struktur

%% PUNKTE
- 
- stehende internetverbindung vorrausgesetzt

- Risiken innerhalb des Projektes
-> Zeitprobleme (auch ausfall der mitarbeiter)
-> Probleme in der Umsetzung/Fehlschlag proof of concepts
\begin{itemize}
\item stehende Internetverbindung vorausgesetzt, Lösung: lokale Speicherung und spätere Synchronisation
\item fortgeschrittene musikalische Erfahrung notwendig
\end{itemize}


\paragraph{Projektintern}

\begin{itemize}

\item -> Zeitprobleme: andere Module nebenbei, Umfang des Projektes zu groß, Lösung: gründliche Zeiteinteilung und kontinuierliches Arbeiten
\item Probleme in der Umsetzung -> Fehlschlagen des PoC, Lösung: Alternativen überlegen

\end{itemize}

% alter text

Technisch bedingte Risiken wie zum Beispiel der Ausfall von Komponenten des verteilten Systems (Datenbanksystem oder Middleware fällt aus) oder die Vorraussetzung einer stehenden Internetverbindung werden in diesem Abschnitt eher weniger behandelt. Es soll mehr darum gehen, welche Faktoren ein mögliches Risiko im Ablauf, in der Bedienung oder in der Vermarktung des Systems darstellen könnten.
Zunächst kann die Notation der MIDI-Kompositionen für einige Benutzer ungewohnt sein. Zwar sollten Töne und Oktaven für jeden Musiker intuitiv verständlich sein, jedoch ist trotz höherem Aufwand eine Notation in Noten oder Tabulaturen gängiger. Dies könnte möglicherweise einige Benutzer davon abschrecken, das System zu nutzen. Andererseits werden Instrumentenspezifische Notationen wie beispielsweise Violin- und Bassschlüssel sowie Tabulaturunterschiede für Bass und Gitarre vermieden und es gibt eine einfache und einheitliche Notation.
Weiterhin kann natürlich das Fehlschlagen der Finanzierung auf lange Sicht ein Problem darstellen. Funktioniert das Geschäftsmodell nicht richtig, werden Verluste gemacht und das gesamte Projekt muss eingestellt werden.


%%%%%%%%%%%%%%%%%%%%%%%%%%%%%%%%%%%%%%%%%%%%%%%%%%%%%%%%%%%%%%%%%%%%%%%%%%%%%

\section{Projektplan}

% Feingranularer
% Stunden schätzen
% genau einteilen wer was macht

% hier aus excel einfügen

%%%%%%%%%%%%%%%%%%%%%%%%%%%%%%%%%%%%%%%%%%%%%%%%%%%%%%%%%%%%%%%%%%%%%%%%%%%%%

\end{document}
